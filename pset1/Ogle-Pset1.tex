% Options for packages loaded elsewhere
\PassOptionsToPackage{unicode}{hyperref}
\PassOptionsToPackage{hyphens}{url}
%
\documentclass[
  12pt,
  landscape]{article}
\usepackage{lmodern}
\usepackage{amssymb,amsmath}
\usepackage{ifxetex,ifluatex}
\ifnum 0\ifxetex 1\fi\ifluatex 1\fi=0 % if pdftex
  \usepackage[T1]{fontenc}
  \usepackage[utf8]{inputenc}
  \usepackage{textcomp} % provide euro and other symbols
\else % if luatex or xetex
  \usepackage{unicode-math}
  \defaultfontfeatures{Scale=MatchLowercase}
  \defaultfontfeatures[\rmfamily]{Ligatures=TeX,Scale=1}
\fi
% Use upquote if available, for straight quotes in verbatim environments
\IfFileExists{upquote.sty}{\usepackage{upquote}}{}
\IfFileExists{microtype.sty}{% use microtype if available
  \usepackage[]{microtype}
  \UseMicrotypeSet[protrusion]{basicmath} % disable protrusion for tt fonts
}{}
\makeatletter
\@ifundefined{KOMAClassName}{% if non-KOMA class
  \IfFileExists{parskip.sty}{%
    \usepackage{parskip}
  }{% else
    \setlength{\parindent}{0pt}
    \setlength{\parskip}{6pt plus 2pt minus 1pt}}
}{% if KOMA class
  \KOMAoptions{parskip=half}}
\makeatother
\usepackage{xcolor}
\IfFileExists{xurl.sty}{\usepackage{xurl}}{} % add URL line breaks if available
\IfFileExists{bookmark.sty}{\usepackage{bookmark}}{\usepackage{hyperref}}
\hypersetup{
  pdftitle={Applications of Econometrics and Data Science Methods Pset1},
  pdfauthor={Jack Ogle},
  hidelinks,
  pdfcreator={LaTeX via pandoc}}
\urlstyle{same} % disable monospaced font for URLs
\usepackage[margin=1in]{geometry}
\usepackage{graphicx,grffile}
\makeatletter
\def\maxwidth{\ifdim\Gin@nat@width>\linewidth\linewidth\else\Gin@nat@width\fi}
\def\maxheight{\ifdim\Gin@nat@height>\textheight\textheight\else\Gin@nat@height\fi}
\makeatother
% Scale images if necessary, so that they will not overflow the page
% margins by default, and it is still possible to overwrite the defaults
% using explicit options in \includegraphics[width, height, ...]{}
\setkeys{Gin}{width=\maxwidth,height=\maxheight,keepaspectratio}
% Set default figure placement to htbp
\makeatletter
\def\fps@figure{htbp}
\makeatother
\setlength{\emergencystretch}{3em} % prevent overfull lines
\providecommand{\tightlist}{%
  \setlength{\itemsep}{0pt}\setlength{\parskip}{0pt}}
\setcounter{secnumdepth}{-\maxdimen} % remove section numbering
\usepackage{dcolumn}
\usepackage{float}

\title{Applications of Econometrics and Data Science Methods Pset1}
\author{Jack Ogle}
\date{}

\begin{document}
\maketitle

Problem 1 (a)

Propose a sequence of random variables that converges in probability to
p.~This sequence must be a function that maps sample sizes (i.e.~natural
numbers) to random variables. Hint: Recall the weak law of large
numbers.

From the Law of Large numbers and the Central Limit Theorem we know that
a sequence of random variables \({\theta_N }\) converges in probability
to \({\theta}\) if for any \({\epsilon} > 0\) and \({\delta} > 0\) there
exists \({N^*} = N^*(\epsilon, \delta)\) such that for all \({N > N^*}\)
\textbackslash{}

Where \(\theta_N\) is a function of \((y_i, X_i, i = 1,...N)\)
\textbackslash{}

\[
Pr(|\theta_N-\theta|<\epsilon) >1-\delta
\] If the limit exists we write that
\[p\lim_{N\to \infty}\theta_N = \theta\] or \$\theta\_N \$ converges in
probability to \(\theta\). And because p is the percentage of COVID-19
infected in Chicago we also know that it has a Bernoulli distribution.
Our end goal is to estimate the number of total test kits needed for the
entire population of Chicago. Therefore we want to estimate the
postitivity rate of Chicago. We can't observe the entire population, but
we can observe a sample. We use the sample to estimate p and we know
that through the law of large numbers and the central limit theorem if
you substitute \(\theta\) for p.~Our sample \(Pn\) will converge in
probability to \(P\). \textbackslash{}

\begin{enumerate}
\def\labelenumi{(\alph{enumi})}
\setcounter{enumi}{1}
\item
\item
\item
\item
\item
\item
\item
\end{enumerate}

\end{document}
